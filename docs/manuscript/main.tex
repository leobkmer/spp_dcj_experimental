% This is samplepaper.tex, a sample chapter demonstrating the
% LLNCS macro package for Springer Computer Science proceedings;
% Version 2.21 of 2022/01/12
%
\documentclass[runningheads]{llncs}
%
%Misc includes
\usepackage{xcolor}
\usepackage{algorithm}
\usepackage{amsmath}
\usepackage{amssymb}
\usepackage{glossaries}

%ILP stuff
\usepackage{fmtcount}
\newcounter{constraintcounter}
\newcounter{domaincounter}
\newenvironment{constraints}{}{}%\setcounter{constraintcounter}{0}}{}
\newenvironment{domains}{\setcounter{domaincounter}{0}}{}
\renewcommand{\theconstraintcounter}{\texttt{C.\padzeroes[2]{\decimal{constraintcounter}}}}
\renewcommand{\thedomaincounter}{\texttt{D.\padzeroes[2]{\decimal{domaincounter}}}}
\newcommand{\cns}{\refstepcounter{constraintcounter}(\theconstraintcounter) }
\newcommand{\dmn}{\refstepcounter{domaincounter}(\thedomaincounter) }
%End ILP stuff

%Comments etc
\newcommand{\comment}[1]{{\color{gray} #1 }}
\newcommand{\todo}[1]{\textbf{\color{red}\large TODO:} \comment{#1}}

\newcommand{\todoLeo}[1]{\textbf{\color{magenta}\large TODO LEONARD:} \comment{#1}}
\newcommand{\todoDany}[1]{\textbf{\color{orange}\large TODO DANY:} \comment{#1}}

%Use this to hide comments upon publishing.
%\renewcommand{\comment}[1]{}
%\renewcommand{\todoLeo}[1]{}
%\renewcommand{\todoDany}[1]{}
%\renewcommand{\todo}[1]{}

%Definitions

\newcommand{\tree}{\Gamma}
\newcommand{\treeedge}{E}
\newcommand{\edt}{\treeedge}
%needed because I sometimes have this in old constraints
\newcommand{\trans}{}
\newcommand{\cis}{}
\newcommand{\genome}[1]{\mathbb{#1}}
\newcommand{\A}{\genome{A}}
\newcommand{\B}{\genome{B}}
\newcommand{\C}{\genome{C}}
\newcommand{\X}{\genome{X}}
\newcommand{\vertices}{V}
\newcommand{\families}{\mathcal F}
\newcommand{\minfam}[1]{\ensuremath{#1_{\text{min}}}}
\newcommand{\maxfam}[1]{\ensuremath{#1_{\text{max}}}}
\newcommand{\pseudocaps}{\mathcal{T}}
\newcommand{\adjacencyedges}{E_\text{adj}}
\newcommand{\extremityedges}{E_{\text{ext}}}
\newcommand{\selfedges}{E_\text{self}}
\newcommand{\ades}{\adjacencyedges}
\newcommand{\exes}{\extremityedges}
\newcommand{\sees}{\selfedges}
\newcommand{\mrd}{\mathcal{MRD}}
\newcommand{\pcmrd}{\mrd^*}
\newcommand{\ggraph}{\mathcal{G}}
\newcommand{\parityvar}{l}


%Glossary defs
\newglossaryentry{pcaps}{name=pseudo-caps, description={}}
\newacronym{mrd}{MRD}{Multi-relational Diagram}

\usepackage[T1]{fontenc}
% T1 fonts will be used to generate the final print and online PDFs,
% so please use T1 fonts in your manuscript whenever possible.
% Other font encondings may result in incorrect characters.
%
\usepackage{graphicx}
% Used for displaying a sample figure. If possible, figure files should
% be included in EPS format.
%
% If you use the hyperref package, please uncomment the following two lines
% to display URLs in blue roman font according to Springer's eBook style:

%\renewcommand\UrlFont{\color{blue}\rmfamily}
%

\begin{document}
%
\title{Reconstructing Rearrangement Phylogenies}
%
%\titlerunning{Abbreviated paper title}
% If the paper title is too long for the running head, you can set
% an abbreviated paper title here
%
\author{First Author\inst{1}\orcidID{0000-1111-2222-3333} \and
Second Author\inst{2,3}\orcidID{1111-2222-3333-4444} \and
Third Author\inst{3}\orcidID{2222--3333-4444-5555}}
%
\authorrunning{F. Author et al.}
% First names are abbreviated in the running head.
% If there are more than two authors, 'et al.' is used.
%
\institute{Princeton University, Princeton NJ 08544, USA \and
Springer Heidelberg, Tiergartenstr. 17, 69121 Heidelberg, Germany
\email{lncs@springer.com}\\
\url{http://www.springer.com/gp/computer-science/lncs} \and
ABC Institute, Rupert-Karls-University Heidelberg, Heidelberg, Germany\\
\email{\{abc,lncs\}@uni-heidelberg.de}}
%
\maketitle              % typeset the header of the contribution
%
\begin{abstract}
The abstract should briefly summarize the contents of the paper in
150--250 words.

\keywords{First keyword  \and Second keyword \and Another keyword.}
\end{abstract}
%
%
%

\section{Introduction}
\section{Background}
\subsection{Problem Formulation}
\section{A new Method}
\subsection{Pre-selecting Adjacencies}
\todoDany{Describe Process}
\subsection{A New ILP Formulation}
\todoLeo{ILP goes here}


\section{Evaluation}
\section{Discussion}


\nocite{DOE-CHA-2021}
\nocite{BOH-BRA-DOE-STO-2021}
\subsubsection{Acknowledgements} Please place your acknowledgments at
the end of the paper, preceded by an unnumbered run-in heading (i.e.
3rd-level heading).

%
% ---- Bibliography ----
%
% BibTeX users should specify bibliography style 'splncs04'.
% References will then be sorted and formatted in the correct style.
%
\bibliographystyle{splncs04}
\bibliography{refs}


\end{document}
