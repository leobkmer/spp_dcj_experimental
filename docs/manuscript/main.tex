% This is samplepaper.tex, a sample chapter demonstrating the
% LLNCS macro package for Springer Computer Science proceedings;
% Version 2.21 of 2022/01/12
%
\documentclass[runningheads]{llncs}
%
%Misc includes
\usepackage{xcolor}
\usepackage{algorithm}
\usepackage{amsmath}
\usepackage{amssymb}
\usepackage{glossaries}

%ILP stuff
\usepackage{fmtcount}
\newcounter{constraintcounter}
\newcounter{domaincounter}
\newenvironment{constraints}{}{}%\setcounter{constraintcounter}{0}}{}
\newenvironment{domains}{\setcounter{domaincounter}{0}}{}
\renewcommand{\theconstraintcounter}{\texttt{C.\padzeroes[2]{\decimal{constraintcounter}}}}
\renewcommand{\thedomaincounter}{\texttt{D.\padzeroes[2]{\decimal{domaincounter}}}}
\newcommand{\cns}{\refstepcounter{constraintcounter}(\theconstraintcounter) }
\newcommand{\dmn}{\refstepcounter{domaincounter}(\thedomaincounter) }
%End ILP stuff

%Comments etc
\newcommand{\comment}[1]{{\color{gray} #1 }}
\newcommand{\todo}[1]{\textbf{\color{red}\large TODO:} \comment{#1}}

\newcommand{\todoLeo}[1]{\textbf{\color{magenta}\large TODO LEONARD:} \comment{#1}}
\newcommand{\todoDany}[1]{\textbf{\color{orange}\large TODO DANY:} \comment{#1}}

%Use this to hide comments upon publishing.
%\renewcommand{\comment}[1]{}
%\renewcommand{\todoLeo}[1]{}
%\renewcommand{\todoDany}[1]{}
%\renewcommand{\todo}[1]{}

%Definitions

\newcommand{\tree}{\Gamma}
\newcommand{\treeedge}{E}
\newcommand{\edt}{\treeedge}
%needed because I sometimes have this in old constraints
\newcommand{\trans}{}
\newcommand{\cis}{}
\newcommand{\genome}[1]{\mathbb{#1}}
\newcommand{\A}{\genome{A}}
\newcommand{\B}{\genome{B}}
\newcommand{\C}{\genome{C}}
\newcommand{\X}{\genome{X}}
\newcommand{\vertices}{V}
\newcommand{\families}{\mathcal F}
\newcommand{\minfam}[1]{\ensuremath{#1_{\text{min}}}}
\newcommand{\maxfam}[1]{\ensuremath{#1_{\text{max}}}}
\newcommand{\pseudocaps}{\mathcal{T}}
\newcommand{\adjacencyedges}{E_\text{adj}}
\newcommand{\extremityedges}{E_{\text{ext}}}
\newcommand{\selfedges}{E_\text{self}}
\newcommand{\ades}{\adjacencyedges}
\newcommand{\exes}{\extremityedges}
\newcommand{\sees}{\selfedges}
\newcommand{\mrd}{\mathcal{MRD}}
\newcommand{\pcmrd}{\mrd^*}
\newcommand{\ggraph}{\mathcal{G}}
\newcommand{\parityvar}{l}


%Glossary defs
\newglossaryentry{pcaps}{name=pseudo-caps, description={}}
\newacronym{mrd}{MRD}{Multi-relational Diagram}

\usepackage[T1]{fontenc}
% T1 fonts will be used to generate the final print and online PDFs,
% so please use T1 fonts in your manuscript whenever possible.
% Other font encondings may result in incorrect characters.
%
\usepackage{graphicx}
% Used for displaying a sample figure. If possible, figure files should
% be included in EPS format.
%
% If you use the hyperref package, please uncomment the following two lines
% to display URLs in blue roman font according to Springer's eBook style:

%\renewcommand\UrlFont{\color{blue}\rmfamily}
%

\begin{document}
%
\title{Reconstructing Rearrangement Phylogenies of Natural Genomes}
%
%\titlerunning{Abbreviated paper title}
% If the paper title is too long for the running head, you can set
% an abbreviated paper title here
%
\author{Leonard Bohnen\"amper\inst{1}\orcidID{0000-0003-4508-0078} \and
    Jens Stoye \inst{1}\orcidID{0000-0002-4656-7155} \and
    Daniel D\"orr \inst{2}\orcidID{0000-0002-3720-6227}
}
%
\authorrunning{L. Bohnenk\"amper et al.}
% First names are abbreviated in the running head.
% If there are more than two authors, 'et al.' is used.
%
\institute{Faculty of Technology and Center for Biotechnology (CeBiTec), Bielefeld University, Germany\\ 
    \email{lbohnenkaemper@techfak.uni-bielefeld.de}, \email{jens.stoye@uni-bielefeld.de}\\
\and
Institute for Medical Biometry and Bioinformatics, Medical Faculty, and Center for Digital Medicine, Heinrich Heine University, Germany\\
\email{daniel.doerr@hhu.de}}

%
\maketitle              % typeset the header of the contribution
%
\begin{abstract}

    We study the classic problem of inferring ancestral genomes under a given phylogeny and from a given set of extant genomes, also known as the \emph{small parsimony problem} (SPP).
    The evolutionary model considered at hand encompasses large scale rearrangements acting on so-called \emph{gene order} sequences and includes segmental gain and loss. Each genome may encompass one or more linear or circular chromosomes, and genes may appear in several copies, without restriction on their genomic location or orientation within the sequence.
    Therefore, our method presumes ancestral copy number estimates, specified either as fixed count, or as range. 
    In case of the latter, our method chooses the copy number within the range that minimizes the total number of large scale rarrangements over the entire phylogeny. 
    
Even under simple evolutionary models, such as the classic character-state model, the SPP is computationally intractable.
However, we give a highly optimized ILP that is able to compute the SPP under our model for sufficiently small phylogenies and gene families. We benchmark our method on simulated phylogenies and discuss its performance in reconstructing gene oders under our broad evolutionary model. 




\keywords{genome rearrangement \and ancestral reconstruction \and small parsimony \and integer linear programming \and double-cut-and-join}
\end{abstract}




\section{Introduction}
\section{Background}

\subsection{Preliminaries}
A \emph{(genomic) marker} $g := (g^\et, g^\eh)$ is a universally unique entity consisting of \emph{marker extremities} tail of $g$, denoted by $g^\et$, and head of $g$, denoted by $g^\eh$. 
A \emph{telomere} $t^\circ$ is a universally unique entity encompassing a single telomeric extremity denoted by ``$\circ$''. 
A genome is a triple $(M \cup \pseudocaps, A, \family)$, with $M$ being its marker set and $\pseudocaps$ its set of telomeres, $A \subset (M \cup \pseudocaps)^2$ the set of telomeres, and $\family: M \to \mathbb N$ a function indicating the family of each marker. 
The set of adjacencies $A$ has the properties that \emph{(i)} for each extremity $\varepsilon \in \bigcup M \cup \pseudocaps$, there exists exactly one adjacency $a \in A$ that contains $\varepsilon$. 
The adjacency set decomposes the set of markers and telomeres into linear or circular components, that we further call \emph{chromosomes}. 
Moreover, \emph{(ii)} each linear chromosome, formed by ordering markers and telomeres based on their adjacencies in $A$, starts and ends with a telomere. 

A \emph{(genome) scaffold} is a quintuple $(M \cup \pseudocaps, A, \family, \multi, \w)$ with 
\begin{itemize}
    \item $M$, $\pseudocaps$, and $\family$ as defined in a genome, 
    \item the set of adjacencies $A$ satisfying that for each $g^\et \in \bigcup A$, there exists also extremity $g^\eh \in \bigcup A$ and vice versa, and each telomeric extremity is used only once, i.e., $\forall \{X, X'\} \subseteq A$, $X \cap X' \cap \pseudocaps = \emptyset$,
    \item copy number function $\multi(i) = [a, b]$ reporting for each family $i$ the permitted range $0 \leq a \leq b$ of its copy number, such that each family, defined as $F_i := \{g \in M \mid f(g) = i\}$, satisfies $|F_i| = b$, and
    \item adjacency weight function $\w: A \to \mathbb R$. % [-1, 1]$.
\end{itemize}

Observe that a genome can give rise to a scaffold, with copy number $\multi$ and weight $\w$ functions arbitrarily defined, but the reverse does not hold true in general.
% not sure we need this:
% However, in the following, we regard genomes as scaffolds with tight copy numbers, that is, the copy number range of each family $i$ corresponds to a point that coincides with the number of its associated markers.
A genome $S' = (M', A', \family')$ is \emph{derived} from a scaffold $S = (M, A, \family, \multi, \w)$, or simply ``$S$-derived'', if \emph{(i)} $M' \subseteq M$, \emph{(ii)} $A' \subseteq A$, \emph{(iii)} 
there exist no two markers $g, g' \in M'$ with $\family'(g) = \family'(g')$ and $\family(g) \neq \family(g')$, and \emph{(iv)} the original copy number constraints are satisfied, i.e., for each family $i$ and the set of markers associated with $i$ in scaffold $S$, i.e., $F_i = \{ g \in M' \mid \family(g) = i\}$, holds true that $|F_i| \in \multi(i)$.
We call a scaffold $S$ \emph{linearizable} if there exists an $S$-derived genome. 
In fact, many scaffolds are not linearizable. 

A \emph{phylogeny} $\tree$ is a connected graph with nodes representing \emph{operational taxonomic units} (OTUs). 
Nodes with degree 1 are termed \emph{tips} of the phylogeny.

General notation
\begin{itemize}
    \item 
    \item Weighted multigraph $G = (\vertices, \edges, \w)$ with edge weight function $\w: \edges \to \mathbb R$
\end{itemize}

Paper-specific notation:

\begin{itemize}
    \item Genomic marker 
    \item Furthermore, we use a function $\extf : \extremities \to \{\et, \eh, \circ\}$ to map extremities to their corresponding kind (tail, head or telomere). 
    \item 
We model family assignments of marker extremities as a function $\family: \extremities \to \mathbb N$ for which holds true that for any marker $g = \{g^\et, g^\eh\}$, $\family(g^\et) = \family(g^\eh)$.  
Function 
$\minmulti_\X : \mathbb N \to \mathbb N$ reports the minimum multiplicity of a gene family in a given genome $\X$, while $\maxmulti_\X: \mathbb N \to \mathbb N$ reports its maximum multiplicity. 
    \item Multirelational diagram $\mrd$ and \emph{capping-free multi-relational diagram} (CFMRD) $\pcmrd$
\end{itemize}

\begin{itemize}
    \item SPP algoirthms such as 
\end{itemize}

\begin{problem}[Weighted CN-constrained degenerate DCJ indel distance]\label{prb:wdeg_dcj}
    Given a weighting scheme $\w : \extremities \times \extremities \to \mathbb R$, some $\alpha \in [0, 1]$, two linearizable degenerate genomes $\A, \B$ with copy number constraints \hl{XX}\todoD{add data structure} and family assignment $\family$, find $\A$-derived genome $\A'$, $\B$-derived genome $\B'$, and $\family$-derived $\{\A',\B'\}$-resolved family assignment $\family'$ that minimize the linear combination
    $$
    (1-\alpha) \cdot \sum_{X \in \A' \cup \B'} -\w(X) + \alpha \cdot \dist_\DCJid(A', B')\,.
    $$
\end{problem}

\begin{problem}[SPP-DCJ]\label{prb:spp_dcj}
    Given a phylogeny $\tree$ and a set of linearizable degenerate genomes $\A_1, \ldots, \A_k$ corresponding to the node set $V(\tree) = \{\A_1, \ldots, \A_k\}$, find genomes $\A'_1 \subseteq \A_1, \ldots, \A'_k \allowbreak \subseteq \A_k$ that minimize the sum of weighted degenerate DCJ indel distances along the edges of $\tree$. 
\end{problem}

\section{A new Method}
\subsection{Pre-selecting Adjacencies}
\todoD{Describe Process}\subsection{A New ILP Formulation}



The algorithm described in the following operates on two levels: 
on the global level, a genome is derived from each GAG, constituting a set of linear or circular chromosomes. 
On the local level, genomes are connected to each other along the branches of the phylogeny. Each branch gives rise to a pairwise comparison by means of the CFMRD. 
In doing so, the selection of adjacencies of a derived genome is propagated from across CFMRDs, thus ensuring global consistency. 

\paragraph{Global level.} 
The global level deals with the setting of adjacencies or telomeres of (ancestral) genomes. For each extremity or pseudo-cap $\nu$, we determine its presence or absence with a binary variable $\ilpvar{g}_\nu$. We require each extremity to occur, except for pseudo-caps (see Constraints~\ref{c:cn}, \ref{c:cn_consistent}). Each extremity is required to be part of exactly one (possibly telomeric) adjacency (\ref{c:genome}), which ensures a properly linearized genome.

\paragraph{Local level.}
The local level deals with each branch of the tree separately, making use of the CFMRD of the genome pair. Since this part is entirely local to the branch in question, we presume that each vertex $v_i$ of the CFMRD has a unique identifier among all other CFMRDs, making all its variables globally unique. In order to limit the range of the general variable $y_{v_i}$, we also assign each vertex a rank $i$ that is local and only unique within the specific CFMRD.

In order to compute decompositions of CFMRDs, we make use of a capping-free formulation for the computation of the pairwise DCJ indel distance~\cite{BOH-2024}. This formulation is based on the following distance formula,

\begin{equation}
	.... TODO.
\end{equation}

The formulation counts cycles $\ilpvar{c}_E$ as well as 6 different types of paths, namely $\ilpvar{p}^{AB},\ilpvar{p}^{Aa},\ilpvar{p}^{Ab},\ilpvar{p}^{ab},\ilpvar{p}^{Ba},\ilpvar{p}^{Bb}$, depending on whether a path ends in a pseudo-cap (uppercase letter) or a lava vertex (lowercase letter) and the genome in which it ends ($A,B$). Each counting variable $\ilpvar{p}^X$ is set by summing up binary report variables $\ilpvar{r}^X_v$ that are set to $1$ once per component on a specific vertex $v$ (see Constraints~\ref{c:beginsum} to~\ref{c:endsum} and~\ref{c:outsum}). These counters are then combined to the terms of the formula in Constraints~\ref{c:begingeq} to~\ref{c:endgeq} and \ref{c:beginform} to~\ref{c:endform}. The constraints for ensuring the reporting variables being set correctly can be found in Tables~\ref{tab:slmcons}, \ref{tab:regv} and~\ref{tab:pcaps}. For a complete description of the ILP the interested reader is referred to~\cite{BOH-2024}.

We make only few major changes in our local section w.r.t.\ the ILP described in~\cite{BOH-2024}. Firstly, we determine whether an adjacency edge $e$ is set ($x_e=1$) by ``inheriting'' this value from the global setting (see~\ref{c:inheritadj}) of the corresponding adjacency. Secondly, we only allow vertices that are part of the genome ($\ilpvar{g}_v=1$) to contribute to the count of components that decrease the formula ($\ilpvar{z}_v=1$), see~\ref{c:inheritz}.
Finally, due to the fact that ancestral genomes are degenerate, the number of possible circular singletons is much higher and listing all candidates might lead to a combinatorial explosion on some input data. We therefore use a new technique for counting circular singletons without listing all candidates, the constraints of which are listed in Table~\ref{tab:csreport}.

A circular singleton manifests in the graph as a cycle of alternating adjacency and indel edges. The idea of the technique is to have a general integer variable $\ilpvar{w}$ that is required to increase at each adjacency edge in a walk of the cycle. There must then be one point in the walk in which it decreases again. Detecting this, one can then report a circular singleton. For this to work, the walk needs a direction. This is accomplished by annotating the vertices with a binary variable $\ilpvar{d}_v$ that ``flips'' across each pair of connected vertices (see~\ref{c:dflip}). We then require $\ilpvar{w}$ to be the same for vertices connected by an indel edge (see~\ref{c:weq}) and for it to increase by $1$ in the direction of the vertex that has $\ilpvar{d}_v=1$ (see~\ref{c:winc}). We require this except when vertices are not connected $(1-\ilpvar{x}_{uv})=0$ or when reporting a circular singleton ($\ilpvar{r}_v=1$ or $\ilpvar{r}_u=1$). In this case, the constraint is automatically fulfilled by adding the maximum length of circular singletons $K$ to the left hand side of the inequation.

\todoL{Insert a description of the formulation here}

\todoLeo{ILP goes here}
\todoLeo{Remove idx stuff and bring back $v_i$ notation}

\begin{algorithm}
\caption{Capping-free Small Parsimony}
\textbf{Objective}

\newcommand{\idx}{\texttt{ix}}
\hspace{0.5cm}\texttt{Maximize} 
\begin{equation*}
    \sum_{\edt \in\tree} (1-\alpha) w_\edt + \alpha f_\edt 
\end{equation*}

\textbf{For each ancestral genome $\X \in \tree$ with $\ggraph(\X) = (\vertices, \adjacencyedges)$}

\begin{constraints}
\begin{tabular}{lcl}
    \cns & $\minfam{F} \leq \sum_{v \in F} g_v \leq \maxfam{F}$ & $\forall F \in \families$\\
    \cns & $g_v = g_u$ & for each pair of nodes $(u, v)$ corresponding to head/tail extremity of the same gene\todoDany{improve}\\
%    \cns & $g_{v} = x^{\X\A}_{v} = x^{\X\B}_{v} = x^{\X\C}_{v}$& $\forall v \in V$ and genomes $\A$, $\B$, $\C$ adjacent to $\X$ in $\Gamma$\\
    \cns & $\sum_{uv \in \adjacencyedges} a_{uv} = g_u$ & $\forall u \in \vertices$\\
\end{tabular}
\end{constraints}


\medskip
\textbf{For each tree edge $(\A,\B):=\edt\in \tree$ with $\pcmrd(\A,\B) = (\vertices\cup\pseudocaps, \adjacencyedges \cup \extremityedges \cup \selfedges)$:}

\begin{constraints}
\begin{tabular}{lcl}

    \cns & $w_\edt = \sum_e \w(e) x_e$\\
	\cns & $f_\edt = n_\edt - c_\edt + q_\edt + s_\edt$\\
	\cns & $n_\edt = 0.5 \sum_{e \in \extremityedges} x_{e}$\\
    \cns & $c_\edt = \sum_{v\in \vertices} r^c_{v}$\\
    \cns & $2q_\edt \geq p^{a\trans b}+ p^{ABa} + p^{ABb} - p^{A\trans B} $\\
    \cns & $p^{a\trans b} = \sum_{v \in \vertices}r^{a\trans b}_{v}$\\
    \cns & $p^{A\trans b} = \sum_{v \in \pseudocaps^A} r^{A\trans b}_e$\\
    \cns & $p^{B\trans a} = \sum_{v\in \pseudocaps^B} r^{B\trans a}_e$\\
    \cns & $p^{A\cis a} = \sum_{v \in \pseudocaps^A} r^{A\cis a}_e$\\
    \cns & $p^{B\cis b} = \sum_{e \in \pseudocaps^B} r^{B\cis b}_e$\\
    \cns & $p^{ABa} \geq p^{B\trans a}$\\
    \cns & $p^{ABa} \geq p^{A\cis a}$\\
    \cns & $p^{ABb} \geq p^{B\cis b}$\\
    \cns & $p^{ABb} \geq p^{A\trans b}$\\
    \cns & $p^{A\trans B} = \sum_{v\in \pseudocaps^A} r^{A\trans B}_v$\\
    \cns & $s_\edt = \sum_{v\in\vertices} r^{s}_{v}$ \\
    \cns & $\sum_{uv\in E_{ext} \cup E_{id} } x_{uv}= g_{\map(u)}$ & $\forall u\in \vertices$  \\
    \cns & $a_{\map(u)\map(v)} = x_{uv}$ & $\forall uv \in \adjacencyedges$ \\
    \cns & $z_{v} \leq g_{\map (v)}$ & $\forall v \in \vertices$ \\
    (\ref{ilp:slmstart}) to (\ref{ilp:slmend})& Shao-Lin-Moret~\cite{SHA-LIN-MOR-2015} constraints& -- see Table~\ref{tab:slmcons}\\
    (\ref{ilp:regvstart}) to (\ref{ilp:regvend})& Reporting for regular vertices& -- see Table~\ref{tab:regv}\\
    (\ref{ilp:pcstart}) to (\ref{ilp:pcend})& Reporting for \gls{pcaps}& -- see Table~\ref{tab:pcaps}\\
    (\ref{ilp:csstart}) to (\ref{ilp:csend})& Reporting circular singletons& --  see Table~\ref{tab:csreport}\\    
\end{tabular}

\end{constraints}
\todoLeo{Put domains here?}
\todoLeo{Fix formatting}
\end{algorithm}

\newcommand{\idx}{}
\begin{table}

\begin{constraints}
\caption{Shao-Lin-Moret constraints.} \label{tab:slmcons}
\begin{tabular}{lcl}
% \cns\label{ilp:slmstart} & $\sum_{uv\in \ades} x_{e} = 1 $ & $\forall u\in \vertices$\\
    \cns\label{ilp:slmstart} & $x_e=x_d$ & for all sibling edges $e,d$\\
    \cns & $y_v + \idx{u}(1-x_{uv}) \geq y_u$ &$\forall uv \in \adjacencyedges\cup \extremityedges$\\
         & $\idx{u}(1-x_{uv})\geq y_u$& $\forall uv \in \selfedges$\\
    \cns\label{ilp:slmend} & $\idx{v}z_v \leq y_v$ & $\forall v\in \vertices\cup\pseudocaps $\\
\end{tabular}
\end{constraints}

\end{table}


\begin{table}

\begin{constraints}
\caption{Reporting for regular vertices.\comment{This is basically exactly the same as in ding, just swap out $r^{ab}$ for $t$ and $\parityvar$ for $r$ and root reports on vertices instead of edges.}} \label{tab:regv}
\begin{tabular}{lcl}
\cns\label{ilp:regvstart} & $\parityvar^\edt_v \leq 1 - x_{uv}$ & $\forall uv \in \selfedges^\A$\\
     & $\parityvar^\edt_v \geq  x_{uv}$ & $\forall uv \in \selfedges^\B$\\
\cns & $\parityvar_v \leq \parityvar_u +  (1-x_{uv})$& $\forall uv \in \extremityedges$\\
& $\parityvar_u \leq \parityvar_v + r_{uv}^{a\trans b} + (1-x_{uv})$& $\forall uv\in E_{adj}^\A,u,v\notin \pseudocaps$\\
& $\parityvar_u \leq \parityvar_v + (1-x_{uv})$& $\forall uv\in E_{adj}^\B,u,v\notin \pseudocaps$\\

\cns & $r_{v}^c \leq z_v$&$\forall v \in \vertices^\A$\\
\cns\label{ilp:regvend} & $r_{u}^{a\trans b} \leq x_{uv}$&$\forall uv\in\selfedges^\A$\\
\end{tabular}
\end{constraints}

\end{table}

\begin{table}

\begin{constraints}
\caption{Reporting for \gls{pcaps}.} \label{tab:pcaps}
\begin{tabular}{lcl}
\cns\label{ilp:pcstart} & $\parityvar_v = 0$ & $\forall v \in \pseudocaps^\A$\\
 & $\parityvar_v = 1$ & $\forall v \in \pseudocaps^\B$\\
\cns & $\parityvar_u \leq \parityvar_v + r_{v}^{A\trans B} + r_{v}^{A\trans b} + (1-x_{uv})$& $\forall uv\in \ades, v \in \pseudocaps^\A$\\
    & $\parityvar_u \leq \parityvar_v + r_{u}^{B\trans a} + (1-x_{uv})$& $\forall uv\in \ades, u\in \pseudocaps^\B$\\
    \cns & $r_{v}^{A\trans B} \leq z_v$&$\forall v \in \pseudocaps^\A$\\
    \cns & $1 -y_v \leq r^{A\trans b}_{v} + r^{A\cis a}_v$ & $v \in \pseudocaps^\A$\\
        & $1 -y_v \leq r^{B\trans a}_{v} + r^{B\cis b}_v$ & $v \in \pseudocaps^\B$\\
    \cns & $y_{v_i} \leq i(1-r_{v}^{R})$&$v\in \pseudocaps^\A,R\in  \{A\trans b, A\cis a\}$\\
     & $y_{v_i} \leq i(1-r_{v}^{R})$&$v\in \pseudocaps^\B,R\in  \{B\trans a, B\cis b\}$\\
    \cns\label{ilp:pcend} & $r_{v}^{A\trans B} \leq b_u$&$\forall uv\in \adjacencyedges, v \in \pseudocaps^A$ \\
    & $r_{v}^{A\trans b} \leq b_u$&$\forall uv\in \adjacencyedges, v \in \pseudocaps^A$ \\
    & $r_{v}^{B\trans a} \leq 1-b_u$&$\forall uv\in \adjacencyedges, v \in \pseudocaps^\B$\\
\end{tabular}
\end{constraints}

\end{table}

\begin{table} \caption{Reporting circular singletons.\comment{(This is what we developed last week.)}}\label{tab:csreport}
\begin{tabular}{lcl}
     \cns\label{ilp:csstart} & $d_u+d_v + x_{uv}\leq 2$ & $\forall uv \in \adjacencyedges\cup \selfedges$  \\
     & $d_u + d_v - x_{uv} \geq 0$ & $\forall uv \in \adjacencyedges\cup \selfedges$  \\
    \cns & $w_u = w_v$ & $\forall uv \in \selfedges$\\
    \cns\label{ilp:csend} &$K (1- x_{uv} + r^{s}_{u} + r^{s}_{v}) + w_v \geq w_u + d_v - d_u $ & $\forall uv\in \adjacencyedges$\\
     %& $K (1- x_{uv} + r^{s}_{v} + r^{s}_{u}) + w_u \geq w_v + d_u - d_v $ & $\forall uv\in \adjacencyedges$\\
\end{tabular}
\end{table}

\todoLeo{Fix inconsistencies with comparison in $\parityvar$ and $z$.}



\section{Evaluation}
\section{Discussion}



\subsubsection{Acknowledgements} Please place your acknowledgments at
the end of the paper, preceded by an unnumbered run-in heading (i.e.
3rd-level heading).

%
% ---- Bibliography ----
%
% BibTeX users should specify bibliography style 'splncs04'.
% References will then be sorted and formatted in the correct style.
%
\bibliographystyle{splncs04}
\bibliography{refs}


\end{document}
