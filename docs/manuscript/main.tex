% This is samplepaper.tex, a sample chapter demonstrating the
% LLNCS macro package for Springer Computer Science proceedings;
% Version 2.21 of 2022/01/12
%
\documentclass[runningheads]{llncs}
%
%Misc includes
\usepackage{xcolor}
\usepackage{algorithm}
\usepackage{amsmath}
\usepackage{amssymb}
\usepackage{glossaries}
\usepackage{placeins}

\usepackage{soul}
\usepackage{todonotes}

%ILP stuff
\usepackage{fmtcount}
\newcounter{constraintcounter}
\newcounter{domaincounter}
\newenvironment{constraints}{}{}%\setcounter{constraintcounter}{0}}{}
\newenvironment{domains}{\setcounter{domaincounter}{0}}{}
\renewcommand{\theconstraintcounter}{\texttt{C.\padzeroes[2]{\decimal{constraintcounter}}}}
\renewcommand{\thedomaincounter}{\texttt{D.\padzeroes[2]{\decimal{domaincounter}}}}
\newcommand{\cns}{\refstepcounter{constraintcounter}(\theconstraintcounter) }
\newcommand{\dmn}{\refstepcounter{domaincounter}(\thedomaincounter) }
%End ILP stuff
%Comments etc
\newcommand{\comment}[1]{{\color{gray} #1 }}
%\newcommand{\todo}[1]{\textbf{\color{red}\large TODO:} \comment{#1}}

\newcommand{\todoL}[1]{\todo[shadow,color=magenta!70,bordercolor=magenta!70]{{\large LEONARD:} #1}}
\newcommand{\todoD}[1]{\todo[shadow,color=orange!70,bordercolor=orange!70]{{\large DANY:} #1}}
\newcommand{\todoLIn}[1]{\todo[color=magenta!70,bordercolor=magenta!70,inline]{{\large LEONARD:} #1}}
\newcommand{\todoDIn}[1]{\todo[color=orange!70,bordercolor=orange!70,inline]{{\large DANY:} #1}}

%Use this to hide comments upon publishing.
%\renewcommand{\comment}[1]{}
%\renewcommand{\todoLeo}[1]{}
%\renewcommand{\todoDany}[1]{}
%\renewcommand{\todo}[1]{}

%Definitions

\newcommand{\dist}{\mathbf{d}}
\newcommand{\DCJid}{\mathrm{DCJ-ID}}
\newcommand{\ddcjid}{d_{\DCJid}}
\newcommand{\tree}{\Gamma}
\newcommand{\treeedge}{E}
\newcommand{\edt}{\treeedge}
%needed because I sometimes have this in old constraints
\newcommand{\trans}{}
\newcommand{\cis}{}
\newcommand{\et}{\textnormal{t}}
\newcommand{\eh}{\textnormal{h}}
\newcommand{\genome}[1]{\mathbb{#1}}
\newcommand{\A}{\genome{A}}
\newcommand{\B}{\genome{B}}
\newcommand{\C}{\genome{C}}
\newcommand{\X}{\genome{X}}
\newcommand{\D}{\genome{D}}
\newcommand{\Lr}{\genome{L}}
\newcommand{\extremities}{\mathcal{E}}
\newcommand{\telomeres}{\mathcal{T}}
\newcommand{\markers}{\mathcal{M}}
%\newcommand{\extremities}{\Epsilon}
\newcommand{\adjacencies}{\mathcal{A}}
\newcommand{\map}{\alpha}
\newcommand{\vertices}{V}
\newcommand{\edges}{E}
\newcommand{\family}{\mathbf{f}}
\newcommand{\extf}{\mathbf{e}}
\newcommand{\multi}{\mathbf{m}}
\newcommand{\minmulti}{\underline{\multi}}
\newcommand{\maxmulti}{\overline{\multi}}
\newcommand{\families}{\mathcal F}
\newcommand{\minfam}[1]{\ensuremath{#1_{\text{min}}}}
\newcommand{\maxfam}[1]{\ensuremath{#1_{\text{max}}}}
\newcommand{\pseudocaps}{\mathcal{T}}
\newcommand{\adjacencyedges}{{\edges}_\text{adj}}
\newcommand{\extremityedges}{{\edges}_{\text{ext}}}
\newcommand{\selfedges}{{\edges}_\text{self}}
\newcommand{\ades}{\adjacencyedges}
\newcommand{\exes}{\extremityedges}
\newcommand{\sees}{\selfedges}
\newcommand{\mrd}{\mathcal{MRD}}
\newcommand{\pcmrd}{\mrd^*}
\newcommand{\ggraph}{\mathcal{G}}
\newcommand{\parityvar}{\ilpvar{l}}
\newcommand{\w}{\mathbf{w}}
\newcommand{\levelsep}[1]{{\color{gray}\emph{#1}\smallskip\hrule\medskip}}
\newcommand{\ilpvar}[1]{{\texttt{#1}}}




%Glossary defs
\newglossaryentry{pcaps}{name=pseudo-caps, description={}}
\newacronym{mrd}{MRD}{Multi-relational Diagram}
\newglossaryentry{linr}{name=linearization, description={}}
\usepackage[T1]{fontenc}
% T1 fonts will be used to generate the final print and online PDFs,
% so please use T1 fonts in your manuscript whenever possible.
% Other font encondings may result in incorrect characters.
%
\usepackage{graphicx}
% Used for displaying a sample figure. If possible, figure files should
% be included in EPS format.
%
% If you use the hyperref package, please uncomment the following two lines
% to display URLs in blue roman font according to Springer's eBook style:

%\renewcommand\UrlFont{\color{blue}\rmfamily}
%

\begin{document}
%
\title{Reconstructing Rearrangement Phylogenies of Natural Genomes}
%
%\titlerunning{Abbreviated paper title}
% If the paper title is too long for the running head, you can set
% an abbreviated paper title here
%
\author{Leonard Bohnen\"amper\inst{1}\orcidID{0000-0003-4508-0078} \and
    Jens Stoye \inst{1}\orcidID{0000-0002-4656-7155} \and
    Daniel D\"orr \inst{2}\orcidID{0000-0002-3720-6227}
}
%
\authorrunning{L. Bohnenk\"amper et al.}
% First names are abbreviated in the running head.
% If there are more than two authors, 'et al.' is used.
%
\institute{Faculty of Technology and Center for Biotechnology (CeBiTec), Bielefeld University, Germany\\ 
    \email{lbohnenkaemper@techfak.uni-bielefeld.de}, \email{jens.stoye@uni-bielefeld.de}\\
\and
Institute for Medical Biometry and Bioinformatics, Medical Faculty, and Center for Digital Medicine, Heinrich Heine University, Germany\\
\email{daniel.doerr@hhu.de}}

%
\maketitle              % typeset the header of the contribution
%
\begin{abstract}

    We study the classic problem of inferring ancestral genomes on a given phylogeny and a given set of extant genomes, also known as \emph{small parsimony problem} (SPP). 
    More specifically, we address a scenario where genomes represent sets of linear or circular successions of oriented genomic markers orders, also known as \emph{gene orders} that evolve along the branches of the phylogeny and are altered by a general evolutionary model that supports a genome rearrangements emulated through the double-cut-and-join (DCJ) operation and by segmental insertions and deletions.

    Even under simple evolutionary models, this problem is computationally intractable, 



\keywords{genome rearrangement \and ancestral reconstruction \and small parsimony \and integer linear programming \and double-cut-and-join}
\end{abstract}




\section{Introduction}
\section{Background}

\subsection{Preliminaries}
A \emph{(genomic) marker} $g := (g^\et, g^\eh)$ is a universally unique entity consisting of \emph{marker extremities} tail of $g$, denoted by $g^\et$, and head of $g$, denoted by $g^\eh$. 
A \emph{telomere} $t^\circ$ is a universally unique entity encompassing a single telomeric extremity denoted by ``$\circ$''. 
A genome is a triple $(M \cup \pseudocaps, A, \family)$, with $M$ being its marker set and $\pseudocaps$ its set of telomeres, $A \subset (M \cup \pseudocaps)^2$ the set of telomeres, and $\family: M \to \mathbb N$ a function indicating the family of each marker. 
The set of adjacencies $A$ has the properties that \emph{(i)} for each extremity $\varepsilon \in \bigcup M \cup \pseudocaps$, there exists exactly one adjacency $a \in A$ that contains $\varepsilon$. 
The adjacency set decomposes the set of markers and telomeres into linear or circular components, that we further call \emph{chromosomes}. 
Moreover, \emph{(ii)} each linear chromosome, formed by ordering markers and telomeres based on their adjacencies in $A$, starts and ends with a telomere. 

A \emph{(genome) scaffold} is a quintuple $(M \cup \pseudocaps, A, \family, \multi, \w)$ with 
\begin{itemize}
    \item $M$, $\pseudocaps$, and $\family$ as defined in a genome, 
    \item the set of adjacencies $A$ satisfying that for each $g^\et \in \bigcup A$, there exists also extremity $g^\eh \in \bigcup A$ and vice versa, and each telomeric extremity is used only once, i.e., $\forall \{X, X'\} \subseteq A$, $X \cap X' \cap \pseudocaps = \emptyset$,
    \item copy number function $\multi(i) = [a, b]$ reporting for each family $i$ the permitted range $0 \leq a \leq b$ of its copy number, such that each family, defined as $F_i := \{g \in M \mid f(g) = i\}$, satisfies $|F_i| = b$, and
    \item adjacency weight function $\w: A \to \mathbb R$. % [-1, 1]$.
\end{itemize}

Observe that a genome can give rise to a scaffold, with copy number $\multi$ and weight $\w$ functions arbitrarily defined, but the reverse does not hold true in general.
% not sure we need this:
% However, in the following, we regard genomes as scaffolds with tight copy numbers, that is, the copy number range of each family $i$ corresponds to a point that coincides with the number of its associated markers.
A genome $S' = (M', A', \family')$ is \emph{derived} from a scaffold $S = (M, A, \family, \multi, \w)$, or simply ``$S$-derived'', if \emph{(i)} $M' \subseteq M$, \emph{(ii)} $A' \subseteq A$, \emph{(iii)} 
there exist no two markers $g, g' \in M'$ with $\family'(g) = \family'(g')$ and $\family(g) \neq \family(g')$, and \emph{(iv)} the original copy number constraints are satisfied, i.e., for each family $i$ and the set of markers associated with $i$ in scaffold $S$, i.e., $F_i = \{ g \in M' \mid \family(g) = i\}$, holds true that $|F_i| \in \multi(i)$.
We call a scaffold $S$ \emph{linearizable} if there exists an $S$-derived genome. 
In fact, many scaffolds are not linearizable. 

A \emph{phylogeny} $\tree$ is a connected graph with nodes representing \emph{operational taxonomic units} (OTUs). 
Nodes with degree 1 are termed \emph{tips} of the phylogeny.

General notation
\begin{itemize}
    \item 
    \item Weighted multigraph $G = (\vertices, \edges, \w)$ with edge weight function $\w: \edges \to \mathbb R$
\end{itemize}

Paper-specific notation:

\begin{itemize}
    \item Genomic marker 
    \item Furthermore, we use a function $\extf : \extremities \to \{\et, \eh, \circ\}$ to map extremities to their corresponding kind (tail, head or telomere). 
    \item 
We model family assignments of marker extremities as a function $\family: \extremities \to \mathbb N$ for which holds true that for any marker $g = \{g^\et, g^\eh\}$, $\family(g^\et) = \family(g^\eh)$.  
Function 
$\minmulti_\X : \mathbb N \to \mathbb N$ reports the minimum multiplicity of a gene family in a given genome $\X$, while $\maxmulti_\X: \mathbb N \to \mathbb N$ reports its maximum multiplicity. 
    \item Multirelational diagram $\mrd$ and \emph{capping-free multi-relational diagram} (CFMRD) $\pcmrd$
\end{itemize}

\begin{itemize}
    \item SPP algoirthms such as 
\end{itemize}

\begin{problem}[Weighted CN-constrained degenerate DCJ indel distance]\label{prb:wdeg_dcj}
    Given a weighting scheme $\w : \extremities \times \extremities \to \mathbb R$, some $\alpha \in [0, 1]$, two linearizable degenerate genomes $\A, \B$ with copy number constraints \hl{XX}\todoD{add data structure} and family assignment $\family$, find $\A$-derived genome $\A'$, $\B$-derived genome $\B'$, and $\family$-derived $\{\A',\B'\}$-resolved family assignment $\family'$ that minimize the linear combination
    $$
    (1-\alpha) \cdot \sum_{X \in \A' \cup \B'} -\w(X) + \alpha \cdot \dist_\DCJid(A', B')\,.
    $$
\end{problem}

\begin{problem}[SPP-DCJ]\label{prb:spp_dcj}
    Given a phylogeny $\tree$ and a set of linearizable degenerate genomes $\A_1, \ldots, \A_k$ corresponding to the node set $V(\tree) = \{\A_1, \ldots, \A_k\}$, find genomes $\A'_1 \subseteq \A_1, \ldots, \A'_k \allowbreak \subseteq \A_k$ that minimize the sum of weighted degenerate DCJ indel distances along the edges of $\tree$. 
\end{problem}

\section{A new Method}
\subsection{Pre-selecting Adjacencies}
\todoD{Describe Process}\subsection{A New ILP Formulation}



The algorithm described in the following operates on two levels: 
on the local level, a genome is derived from each GAG, constituting a set of linear or circular chromosomes. 
On the global level, genomes are connected to each other along the branches of the phylogeny. Each branch gives rise to a pairwise comparison by means of the CFMRD. 
In doing so, the selection of adjacencies of a derived genome is propagated from across CFMRDs, thus ensuring global consistency. 

\paragraph{Local level.} 
Constraint \ref{c:cn} implements the user-provided copy number requirement. 
In doing so, we define the set of tail extremities of markers from family $i$ in GAG $\ggraph(\X)$ as $F_i = \{ v \mid v \in V: \family(v) = i \text{ and } \extf(v) = \et\}$ and iterate over the associated $\ilpvar{g}$ variables of its members to bound the number of active markers within the given range. 
The choice of counting tail extremities is arbitrary, however, the subsequent constraint (\ref{c:cn_consistent}) ensures that the assignment of $\ilpvar{g}$ variables for both extremities of a markers is consistent. 
At last, Constraint \ref{c:genome} makes sure that a valid genome is derived from the GAG when solving the ILP, by enforcing that each extremity node of $\ggraph(\X)$ is incident to at most one adjacency edge. 

\paragraph{Global level.}

When comparing pairs of genomes within a CFMRD, we make use of a capping-free formulation for the computation of the pairwise DCJ indel distance~\cite{BOH-2023}.


\todoL{Insert a description of the formulation here}

%\todoLIn{Remove idx stuff and bring back $v_i$ notation} --doneLB

\begin{algorithm}[tbh]
\caption{Capping-free Small Parsimony}
\textbf{Objective}

\newcommand{\idx}{\texttt{ix}}
\hspace{0.5cm}\texttt{Minimize} 
\begin{equation*}
    \sum_{\edt \in\tree} (\alpha-1) \ilpvar{w}_\edt + \alpha \ilpvar{f}_\edt 
\end{equation*}

\levelsep{Global level}

\textbf{For each genome $\X$ of phylogeny $\tree$ with extremities $\extremities(\X)$, markers $\markers(\X)$ and adjacencies $\adjacencies(\X)$.}

\begin{constraints}
\begin{tabular}{lcl}
    \cns \label{c:cn} & $\ilpvar{g}_\nu = 1$& $\nu$ not a pseudo-cap\\
    \cns \label{c:cn_consistent}& $\ilpvar{g}_\nu = \ilpvar{g}_\mu$ & with $(\nu,\mu) \in \markers(\X)$\\% \in \markers (\X)$\\
%    \cns & $g_{v} = x^{\X\A}_{v} = x^{\X\B}_{v} = x^{\X\C}_{v}$& $\forall v \in V$ and genomes $\A$, $\B$, $\C$ adjacent to $\X$ in $\Gamma$\\
    \cns \label{c:genome}& $\sum_{\nu\mu \in \adjacencies (\X)} \ilpvar{a}_{\nu\mu} = \ilpvar{g}_\mu$ & $\forall \mu \in \extremities (\X)$\\
\end{tabular}
\end{constraints}

\medskip
\levelsep{Local level}

\textbf{For each branch $(\A,\B):=\edt\in \tree$ with $\pcmrd(\A,\B) = (\vertices\cup\pseudocaps, \adjacencyedges \cup \extremityedges \cup \selfedges)$:}

\begin{constraints}
\begin{tabular}{lcl}

    \cns\label{c:beginform} & $\ilpvar{w}_\edt = \sum_{e \in \adjacencyedges} \w(e) x_e$\\
	\cns & $\ilpvar{f}_\edt = \ilpvar{n}_\edt - \ilpvar{c}_\edt + \ilpvar{q}_\edt + \ilpvar{s}_\edt$\\
    \cns & $\ilpvar{n}_\edt = \frac{1}{2} \sum_{e \in \extremityedges} \ilpvar{x}_{e}$\\
    \cns & $\ilpvar{c}_\edt = \sum_{v\in \vertices} \ilpvar{r}^c_v$\\
    \cns\label{c:endform} & $2\ilpvar{q}_\edt \geq \ilpvar{p}^{a\trans b}_\edt+ \ilpvar{p}^{ABa}_\edt + \ilpvar{p}^{ABb}_\edt - \ilpvar{p}^{A\trans B}_\edt $\\
    \cns\label{c:beginsum} & $\ilpvar{p}^{a\trans b}_\edt = \sum_{v \in \vertices}\ilpvar{r}^{a\trans b}_v$\\
    \cns & $\ilpvar{p}^{A\trans b}_\edt = \sum_{v \in \pseudocaps^A} \ilpvar{r}^{A\trans b}_e$\\
    \cns & $\ilpvar{p}^{B\trans a}_\edt = \sum_{v\in \pseudocaps^B} \ilpvar{r}^{B\trans a}_e$\\
    \cns & $\ilpvar{p}^{A\cis a}_\edt = \sum_{v \in \pseudocaps^A} \ilpvar{r}^{A\cis a}_e$\\
    \cns\label{c:endsum} & $\ilpvar{p}^{B\cis b}_\edt = \sum_{e \in \pseudocaps^B} \ilpvar{r}^{B\cis b}_e$\\
    \cns\label{c:begingeq} & $\ilpvar{p}^{ABa}_\edt \geq p^{B\trans a}_\edt$\\
    \cns & $\ilpvar{p}^{ABa}_\edt \geq p^{A\cis a}_\edt$\\
    \cns & $\ilpvar{p}^{ABb}_\edt \geq p^{B\cis b}_\edt$\\
    \cns\label{c:endgeq} & $\ilpvar{p}^{ABb}_\edt \geq p^{A\trans b}_\edt$\\
    \cns\label{c:outsum} & $\ilpvar{p}^{A\trans B}_\edt = \sum_{v\in \pseudocaps^A} \ilpvar{r}^{A\trans B}_v$\\
    \cns & $\ilpvar{s}_\edt = \sum_{v\in\vertices} \ilpvar{r}^{s}_{v}$ \\
    \cns\label{c:inheritext} & $\sum_{uv\in E_{ext} \cup E_{id} } \ilpvar{x}_{uv}= \ilpvar{g}_{\map(u)}$ & $\forall u\in \vertices, u\notin \pseudocaps$  \\
    \cns\label{c:inheritadj} & $\ilpvar{a}_{\map(u)\map(v)} = \ilpvar{x}_{uv}$ & $\forall uv \in \adjacencyedges$ \\
    \cns\label{c:inheritz} & $\ilpvar{z}_{v} \leq \ilpvar{g}_{\map (v)}$ & $\forall v \in \vertices$ \\
    (\ref{ilp:slmstart}) to (\ref{ilp:slmend})& Shao-Lin-Moret~\cite{SHA-LIN-MOR-2015} constraints& -- see Table~\ref{tab:slmcons}\\
    (\ref{ilp:regvstart}) to (\ref{ilp:regvend})& Reporting for regular vertices& -- see Table~\ref{tab:regv}\\
    (\ref{ilp:pcstart}) to (\ref{ilp:pcend})& Reporting for \gls{pcaps}& -- see Table~\ref{tab:pcaps}\\
    (\ref{ilp:csstart}) to (\ref{ilp:csend})& Reporting circular singletons& --  see Table~\ref{tab:csreport}\\    
\end{tabular}

\end{constraints}
\todoLIn{Put domains here?}
\todoLIn{Fix formatting}
\end{algorithm}

\FloatBarrier

\begin{table}

\begin{constraints}
\caption{Shao-Lin-Moret constraints.} \label{tab:slmcons}
\begin{tabular}{lcl}
% \cns\label{ilp:slmstart} & $\sum_{uv\in \ades} x_{e} = 1 $ & $\forall u\in \vertices$\\
    \cns\label{ilp:slmstart} & $\ilpvar{x}_e=\ilpvar{x}_d$ & for all sibling edges $e,d$\\
    \cns & $\ilpvar{y}_{v_i} + j(1-\ilpvar{x}_{u_jv_i}) \geq \ilpvar{y}_{u_j}$ &$\forall u_jv_i \in \adjacencyedges\cup \extremityedges$\\
         & $j(1-\ilpvar{x}_{u_jv_i})\geq \ilpvar{y}_{u_j}$& $\forall u_jv_i \in \selfedges$\\
    \cns\label{ilp:slmend} & $i\ilpvar{z}_{v_i} \leq \ilpvar{y}_{v_i}$ & $\forall v\in \vertices\cup\pseudocaps $\\
\end{tabular}
\end{constraints}

\end{table}


\begin{table}

\begin{constraints}
\caption{Reporting for regular vertices.\comment{This is basically exactly the same as in ding, just swap out $r^{ab}$ for $t$ and $\parityvar$ for $r$ and root reports on vertices instead of edges.}} \label{tab:regv}
\begin{tabular}{lcl}
\cns\label{ilp:regvstart} & $\parityvar^\edt_v \leq 1 - \ilpvar{x}_{uv}$ & $\forall uv \in \selfedges^\A$\\
     & $\parityvar^\edt_v \geq  \ilpvar{x}_{uv}$ & $\forall uv \in \selfedges^\B$\\
\cns & $\parityvar_v \leq \parityvar_u +  (1-\ilpvar{x}_{uv})$& $\forall uv \in \extremityedges$\\
& $\parityvar_u \leq \parityvar_v + \ilpvar{r}_{uv}^{a\trans b} + (1-\ilpvar{x}_{uv})$& $\forall uv\in E_{adj}^\A,u,v\notin \pseudocaps$\\
& $\parityvar_u \leq \parityvar_v + (1-\ilpvar{x}_{uv})$& $\forall uv\in E_{adj}^\B,u,v\notin \pseudocaps$\\

\cns & $\ilpvar{r}_{v}^c \leq \ilpvar{z}_v$&$\forall v \in \vertices^\A$\\
\cns\label{ilp:regvend} & $\ilpvar{r}_{u}^{a\trans b} \leq \ilpvar{x}_{uv}$&$\forall uv\in\selfedges^\A$\\
\end{tabular}
\end{constraints}

\end{table}

\begin{table}

\begin{constraints}
\caption{Reporting for \gls{pcaps}.} \label{tab:pcaps}
\begin{tabular}{lcl}
\cns\label{ilp:pcstart} & $\parityvar_v = 0$ & $\forall v \in \pseudocaps^\A$\\
 & $\parityvar_v = 1$ & $\forall v \in \pseudocaps^\B$\\
\cns & $\parityvar_u \leq \parityvar_v + \ilpvar{r}_{v}^{A\trans B} + \ilpvar{r}_{v}^{A\trans b} + (1-\ilpvar{x}_{uv})$& $\forall uv\in \ades, v \in \pseudocaps^\A$\\
    & $\parityvar_u \leq \parityvar_v + \ilpvar{r}_{u}^{B\trans a} + (1-\ilpvar{x}_{uv})$& $\forall uv\in \ades, u\in \pseudocaps^\B$\\
    \cns & $\ilpvar{r}_{v}^{A\trans B} \leq \ilpvar{z}_v$&$\forall v \in \pseudocaps^\A$\\
    \cns & $1 -\ilpvar{y}_v \leq \ilpvar{r}^{A\trans b}_{v} + \ilpvar{r}^{A\cis a}_v$ & $v \in \pseudocaps^\A$\\
        & $1 -\ilpvar{y}_v \leq \ilpvar{r}^{B\trans a}_{v} + \ilpvar{r}^{B\cis b}_v$ & $v \in \pseudocaps^\B$\\
    \cns & $\ilpvar{y}_{v_i} \leq i(1-\ilpvar{r}_{v}^{R})$&$v\in \pseudocaps^\A,R\in  \{A\trans b, A\cis a\}$\\
     & $\ilpvar{y}_{v_i} \leq i(1-\ilpvar{r}_{v}^{R})$&$v\in \pseudocaps^\B,R\in  \{B\trans a, B\cis b\}$\\
    \cns\label{ilp:pcend} & $\ilpvar{r}_{v}^{A\trans B} \leq \parityvar_u$&$\forall uv\in \adjacencyedges, v \in \pseudocaps^A$ \\
    & $\ilpvar{r}_{v}^{A\trans b} \leq \parityvar_u$&$\forall uv\in \adjacencyedges, v \in \pseudocaps^A$ \\
    & $\ilpvar{r}_{v}^{B\trans a} \leq 1-\parityvar_u$&$\forall uv\in \adjacencyedges, v \in \pseudocaps^\B$\\
\end{tabular}
\end{constraints}

\end{table}

\begin{table} \caption{Reporting circular singletons.\comment{(This is what we developed last week.)}}\label{tab:csreport}
\begin{tabular}{lcl}
     \cns\label{ilp:csstart}\label{c:dflip} & $\ilpvar{d}_u+\ilpvar{d}_v + \ilpvar{x}_{uv}\leq 2$ & $\forall uv \in \adjacencyedges\cup \selfedges$  \\
     & $\ilpvar{d}_u + \ilpvar{d}_v - \ilpvar{x}_{uv} \geq 0$ & $\forall uv \in \adjacencyedges\cup \selfedges$  \\
    \cns\label{c:weq} & $\ilpvar{w}_u = \ilpvar{w}_v$ & $\forall uv \in \selfedges$\\
    \cns\label{ilp:csend}\label{c:winc} &$K (1- \ilpvar{x}_{uv} + \ilpvar{r}^{s}_{u} + \ilpvar{r}^{s}_{v}) + \ilpvar{w}_v \geq \ilpvar{w}_u + \ilpvar{d}_v - \ilpvar{d}_u $ & $\forall uv\in \adjacencyedges$\\
     %& $K (1- x_{uv} + r^{s}_{v} + r^{s}_{u}) + w_u \geq w_v + d_u - d_v $ & $\forall uv\in \adjacencyedges$\\
\end{tabular}
\end{table}

\todoL{Fix inconsistencies with comparison in $\parityvar$ and $z$.}



\section{Evaluation}
\section{Discussion}



\subsubsection{Acknowledgements} Please place your acknowledgments at
the end of the paper, preceded by an unnumbered run-in heading (i.e.
3rd-level heading).

%
% ---- Bibliography ----
%
% BibTeX users should specify bibliography style 'splncs04'.
% References will then be sorted and formatted in the correct style.
%
\bibliographystyle{splncs04}
\bibliography{refs}


\end{document}
