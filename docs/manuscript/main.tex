% This is samplepaper.tex, a sample chapter demonstrating the
% LLNCS macro package for Springer Computer Science proceedings;
% Version 2.21 of 2022/01/12
%
\documentclass[runningheads]{llncs}
%
%Misc includes
\usepackage{xcolor}
\usepackage{algorithm}
\usepackage{amsmath}
\usepackage{amssymb}
\usepackage{glossaries}

%ILP stuff
\usepackage{fmtcount}
\newcounter{constraintcounter}
\newcounter{domaincounter}
\newenvironment{constraints}{}{}%\setcounter{constraintcounter}{0}}{}
\newenvironment{domains}{\setcounter{domaincounter}{0}}{}
\renewcommand{\theconstraintcounter}{\texttt{C.\padzeroes[2]{\decimal{constraintcounter}}}}
\renewcommand{\thedomaincounter}{\texttt{D.\padzeroes[2]{\decimal{domaincounter}}}}
\newcommand{\cns}{\refstepcounter{constraintcounter}(\theconstraintcounter) }
\newcommand{\dmn}{\refstepcounter{domaincounter}(\thedomaincounter) }
%End ILP stuff

%Comments etc
\newcommand{\comment}[1]{{\color{gray} #1 }}
\newcommand{\todo}[1]{\textbf{\color{red}\large TODO:} \comment{#1}}

\newcommand{\todoLeo}[1]{\textbf{\color{magenta}\large TODO LEONARD:} \comment{#1}}
\newcommand{\todoDany}[1]{\textbf{\color{orange}\large TODO DANY:} \comment{#1}}

%Use this to hide comments upon publishing.
%\renewcommand{\comment}[1]{}
%\renewcommand{\todoLeo}[1]{}
%\renewcommand{\todoDany}[1]{}
%\renewcommand{\todo}[1]{}

%Definitions

\newcommand{\tree}{\Gamma}
\newcommand{\treeedge}{E}
\newcommand{\edt}{\treeedge}
%needed because I sometimes have this in old constraints
\newcommand{\trans}{}
\newcommand{\cis}{}
\newcommand{\genome}[1]{\mathbb{#1}}
\newcommand{\A}{\genome{A}}
\newcommand{\B}{\genome{B}}
\newcommand{\C}{\genome{C}}
\newcommand{\X}{\genome{X}}
\newcommand{\vertices}{V}
\newcommand{\families}{\mathcal F}
\newcommand{\minfam}[1]{\ensuremath{#1_{\text{min}}}}
\newcommand{\maxfam}[1]{\ensuremath{#1_{\text{max}}}}
\newcommand{\pseudocaps}{\mathcal{T}}
\newcommand{\adjacencyedges}{E_\text{adj}}
\newcommand{\extremityedges}{E_{\text{ext}}}
\newcommand{\selfedges}{E_\text{self}}
\newcommand{\ades}{\adjacencyedges}
\newcommand{\exes}{\extremityedges}
\newcommand{\sees}{\selfedges}
\newcommand{\mrd}{\mathcal{MRD}}
\newcommand{\pcmrd}{\mrd^*}
\newcommand{\ggraph}{\mathcal{G}}
\newcommand{\parityvar}{l}


%Glossary defs
\newglossaryentry{pcaps}{name=pseudo-caps, description={}}
\newacronym{mrd}{MRD}{Multi-relational Diagram}

\usepackage[T1]{fontenc}
% T1 fonts will be used to generate the final print and online PDFs,
% so please use T1 fonts in your manuscript whenever possible.
% Other font encondings may result in incorrect characters.
%
\usepackage{graphicx}
% Used for displaying a sample figure. If possible, figure files should
% be included in EPS format.
%
% If you use the hyperref package, please uncomment the following two lines
% to display URLs in blue roman font according to Springer's eBook style:

%\renewcommand\UrlFont{\color{blue}\rmfamily}
%

\begin{document}
%
\title{Reconstructing Rearrangement Phylogenies of Natural Genomes}
%
%\titlerunning{Abbreviated paper title}
% If the paper title is too long for the running head, you can set
% an abbreviated paper title here
%
\author{Leonard Bohnen\"amper\inst{1}\orcidID{0000-0003-4508-0078} \and
    Jens Stoye \inst{1}\orcidID{0000-0002-4656-7155} \and
    Daniel D\"orr \inst{2}\orcidID{0000-0002-3720-6227}
}
%
\authorrunning{L. Bohnenk\"amper et al.}
% First names are abbreviated in the running head.
% If there are more than two authors, 'et al.' is used.
%
\institute{Faculty of Technology and Center for Biotechnology (CeBiTec), Bielefeld University, Germany\\ 
    \email{lbohnenkaemper@techfak.uni-bielefeld.de}, \email{jens.stoye@uni-bielefeld.de}\\
\and
Institute for Medical Biometry and Bioinformatics, Medical Faculty, and Center for Digital Medicine, Heinrich Heine University, Germany\\
\email{daniel.doerr@hhu.de}}

%
\maketitle              % typeset the header of the contribution
%
\begin{abstract}

    We study the classic problem of inferring ancestral genomes under a given phylogeny and from a given set of extant genomes, also known as the \emph{small parsimony problem} (SPP).
    The evolutionary model considered at hand encompasses large scale rearrangements acting on so-called \emph{gene order} sequences and includes segmental gain and loss. Each genome may encompass one or more linear or circular chromosomes, and genes may appear in several copies, without restriction on their genomic location or orientation within the sequence.
    Therefore, our method presumes ancestral copy number estimates, specified either as fixed count, or as range. 
    In case of the latter, our method chooses the copy number within the range that minimizes the total number of large scale rarrangements over the entire phylogeny. 
    
Even under simple evolutionary models, such as the classic character-state model, the SPP is computationally intractable.
However, we give a highly optimized ILP that is able to compute the SPP under our model for sufficiently small phylogenies and gene families. We benchmark our method on simulated phylogenies and discuss its performance in reconstructing gene oders under our broad evolutionary model. 




\keywords{genome rearrangement \and ancestral reconstruction \and small parsimony \and integer linear programming \and double-cut-and-join}
\end{abstract}




\section{Introduction}
\section{Background}

\subsection{Preliminaries}
A \emph{phylogeny} $\tree$ is a connected graph with nodes representing \emph{operational taxonomic units} (OTUs). 
Nodes with degree 1 are termed \emph{tips} of the phylogeny.
A \emph{genome scaffold} of an OTU $\A$ is a quintuple $(M \cup \pseudocaps, A, \family, \multi, \w)$ with 
\begin{itemize}
    \item $M$ being the marker set of $\A$ and $\pseudocaps$ its set of telomeres,
    \item $A$ its set of unique adjacencies satisfying 
        \emph{(i)} $\forall g^\et \in \bigcup A$, there exists also extremity $g^\eh \in \bigcup A$ and vice versa, and 
        \emph{(ii)} each telomeric extremity is used only once: $\forall \{X, X'\} \subseteq A$, $X \cap X' \cap \pseudocaps = \emptyset$. 
    \item $\family: M \to \mathbb N$ a function indicating the gene family of each marker of $\A$,
    \item a function $\multi(i) = [a, b]$ that reports for each family $i$ the permitted range $0 \leq a \leq b$ of its copy number, such that each family $F_i = \{m \in M \mid f(m) = i\}$ satisfies $|F_i| = b$, 
    \item and adjacency weight function $\w: A \to \mathbb R$.
\end{itemize}
% \begin{enumerate}
% \end{enumerate}
A genome is a triple $(M \cup \pseudocaps, A, \family)$ for which the following holds true:
\begin{itemize}
    \item The extremities of each marker $m = (m^\et, m^\eh) \in M$ occur in exactly one adjacency, i.e., $|\{ a \in A \mid m^\et \in a\}| = |\{ a \in A \mid m^\eh \in a\}| = 1$,
    \item 
\end{itemize}
Note that a genome is also a scaffold, but the reverse does not hold true in general. 

A genome $A'$ is \emph{$A$-derived} from degenerate genome $A$ (or simply ``derived from $A$'') if $A' \subseteq A$ and $\bigcup A' \setminus \pseudocaps = \bigcup A \setminus \pseudocaps$. 
Conversely, a degenerate genome $A$ is \emph{linearizable} if there exists an $A$-derived genome. 
In fact, many degenerate genomes are not linearizable. 

General notation
\begin{itemize}
    \item 
    \item Weighted multigraph $G = (\vertices, \edges, \w)$ with edge weight function $\w: \edges \to \mathbb R$
\end{itemize}

Paper-specific notation:

\begin{itemize}
    \item Genomic marker 
    \item Furthermore, we use a function $\extf : \extremities \to \{\et, \eh, \circ\}$ to map extremities to their corresponding kind (tail, head or telomere). 
    \item Genome $\X$
    \item \hl{Genome adjacency graph}\todoD{replace with degenerate genome} $\ggraph(\X) = (\vertices, \edges, \family)$ 
    \item 
We model family assignments of marker extremities as a function $\family: \extremities \to \mathbb N$ for which holds true that for any marker $g = \{g^\et, g^\eh\}$, $\family(g^\et) = \family(g^\eh)$.  
Function 
$\minmulti_\X : \mathbb N \to \mathbb N$ reports the minimum multiplicity of a gene family in a given genome $\X$, while $\maxmulti_\X: \mathbb N \to \mathbb N$ reports its maximum multiplicity. 
    \item Multirelational diagram $\mrd$ and \emph{capping-free multi-relational diagram} (CFMRD) $\pcmrd$
\end{itemize}

\begin{itemize}
    \item SPP algoirthms such as 
\end{itemize}

\begin{problem}[Weighted CN-constrained degenerate DCJ indel distance]\label{prb:wdeg_dcj}
    Given a weighting scheme $\w : \extremities \times \extremities \to \mathbb R$, some $\alpha \in [0, 1]$, two linearizable degenerate genomes $\A, \B$ with copy number constraints \hl{XX}\todoD{add data structure} and family assignment $\family$, find $\A$-derived genome $\A'$, $\B$-derived genome $\B'$, and $\family$-derived $\{\A',\B'\}$-resolved family assignment $\family'$ that minimize the linear combination
    $$
    (1-\alpha) \cdot \sum_{X \in \A' \cup \B'} -\w(X) + \alpha \cdot \dist_\DCJid(A', B')\,.
    $$
\end{problem}

\begin{problem}[SPP-DCJ]\label{prb:spp_dcj}
    Given a phylogeny $\tree$ and a set of linearizable degenerate genomes $\A_1, \ldots, \A_k$ corresponding to the node set $V(\tree) = \{\A_1, \ldots, \A_k\}$, find genomes $\A'_1 \subseteq \A_1, \ldots, \A'_k \allowbreak \subseteq \A_k$ that minimize the sum of weighted degenerate DCJ indel distances along the edges of $\tree$. 
\end{problem}

\section{A new Method}
\subsection{Pre-selecting Adjacencies}
\todoD{Describe Process}\subsection{A New ILP Formulation}



The algorithm described in the following operates on two levels: 
on the local level, a genome is derived from each GAG, constituting a set of linear or circular chromosomes. 
On the global level, genomes are connected to each other along the branches of the phylogeny. Each branch gives rise to a pairwise comparison by means of the CFMRD. 
In doing so, the selection of adjacencies of a derived genome is propagated from across CFMRDs, thus ensuring global consistency. 

\paragraph{Local level.} 
Constraint \ref{c:cn} implements the user-provided copy number requirement. 
In doing so, we define the set of tail extremities of markers from family $i$ in GAG $\ggraph(\X)$ as $F_i = \{ v \mid v \in V: \family(v) = i \text{ and } \extf(v) = \et\}$ and iterate over the associated $\ilpvar{g}$ variables of its members to bound the number of active markers within the given range. 
The choice of counting tail extremities is arbitrary, however, the subsequent constraint (\ref{c:cn_consistent}) ensures that the assignment of $\ilpvar{g}$ variables for both extremities of a markers is consistent. 
At last, Constraint \ref{c:genome} makes sure that a valid genome is derived from the GAG when solving the ILP, by enforcing that each extremity node of $\ggraph(\X)$ is incident to at most one adjacency edge. 

\paragraph{Global level.}

When comparing pairs of genomes within a CFMRD, we make use of a capping-free formulation for the computation of the pairwise DCJ indel distance~\cite{BOH-2023}.


\todoL{Insert a description of the formulation here}

\todoLeo{ILP goes here}
\todoLeo{Remove idx stuff and bring back $v_i$ notation}

\begin{algorithm}
\caption{Capping-free Small Parsimony}
\textbf{Objective}

\newcommand{\idx}{\texttt{ix}}
\hspace{0.5cm}\texttt{Maximize} 
\begin{equation*}
    \sum_{\edt \in\tree} (1-\alpha) w_\edt + \alpha f_\edt 
\end{equation*}

\textbf{For each ancestral genome $\X \in \tree$ with $\ggraph(\X) = (\vertices, \adjacencyedges)$}

\begin{constraints}
\begin{tabular}{lcl}
    \cns & $\minfam{F} \leq \sum_{v \in F} g_v \leq \maxfam{F}$ & $\forall F \in \families$\\
    \cns & $g_v = g_u$ & for each pair of nodes $(u, v)$ corresponding to head/tail extremity of the same gene\todoDany{improve}\\
%    \cns & $g_{v} = x^{\X\A}_{v} = x^{\X\B}_{v} = x^{\X\C}_{v}$& $\forall v \in V$ and genomes $\A$, $\B$, $\C$ adjacent to $\X$ in $\Gamma$\\
    \cns & $\sum_{uv \in \adjacencyedges} a_{uv} = g_u$ & $\forall u \in \vertices$\\
\end{tabular}
\end{constraints}


\medskip
\textbf{For each tree edge $(\A,\B):=\edt\in \tree$ with $\pcmrd(\A,\B) = (\vertices\cup\pseudocaps, \adjacencyedges \cup \extremityedges \cup \selfedges)$:}

\begin{constraints}
\begin{tabular}{lcl}

    \cns & $w_\edt = \sum_e \w(e) x_e$\\
	\cns & $f_\edt = n_\edt - c_\edt + q_\edt + s_\edt$\\
	\cns & $n_\edt = 0.5 \sum_{e \in \extremityedges} x_{e}$\\
    \cns & $c_\edt = \sum_{v\in \vertices} r^c_{v}$\\
    \cns & $2q_\edt \geq p^{a\trans b}+ p^{ABa} + p^{ABb} - p^{A\trans B} $\\
    \cns & $p^{a\trans b} = \sum_{v \in \vertices}r^{a\trans b}_{v}$\\
    \cns & $p^{A\trans b} = \sum_{v \in \pseudocaps^A} r^{A\trans b}_e$\\
    \cns & $p^{B\trans a} = \sum_{v\in \pseudocaps^B} r^{B\trans a}_e$\\
    \cns & $p^{A\cis a} = \sum_{v \in \pseudocaps^A} r^{A\cis a}_e$\\
    \cns & $p^{B\cis b} = \sum_{e \in \pseudocaps^B} r^{B\cis b}_e$\\
    \cns & $p^{ABa} \geq p^{B\trans a}$\\
    \cns & $p^{ABa} \geq p^{A\cis a}$\\
    \cns & $p^{ABb} \geq p^{B\cis b}$\\
    \cns & $p^{ABb} \geq p^{A\trans b}$\\
    \cns & $p^{A\trans B} = \sum_{v\in \pseudocaps^A} r^{A\trans B}_v$\\
    \cns & $s_\edt = \sum_{v\in\vertices} r^{s}_{v}$ \\
    \cns & $\sum_{uv\in E_{ext} \cup E_{id} } x_{uv}= g_{\map(u)}$ & $\forall u\in \vertices$  \\
    \cns & $a_{\map(u)\map(v)} = x_{uv}$ & $\forall uv \in \adjacencyedges$ \\
    \cns & $z_{v} \leq g_{\map (v)}$ & $\forall v \in \vertices$ \\
    (\ref{ilp:slmstart}) to (\ref{ilp:slmend})& Shao-Lin-Moret~\cite{SHA-LIN-MOR-2015} constraints& -- see Table~\ref{tab:slmcons}\\
    (\ref{ilp:regvstart}) to (\ref{ilp:regvend})& Reporting for regular vertices& -- see Table~\ref{tab:regv}\\
    (\ref{ilp:pcstart}) to (\ref{ilp:pcend})& Reporting for \gls{pcaps}& -- see Table~\ref{tab:pcaps}\\
    (\ref{ilp:csstart}) to (\ref{ilp:csend})& Reporting circular singletons& --  see Table~\ref{tab:csreport}\\    
\end{tabular}

\end{constraints}
\todoLeo{Put domains here?}
\todoLeo{Fix formatting}
\end{algorithm}

\newcommand{\idx}{}
\begin{table}

\begin{constraints}
\caption{Shao-Lin-Moret constraints.} \label{tab:slmcons}
\begin{tabular}{lcl}
% \cns\label{ilp:slmstart} & $\sum_{uv\in \ades} x_{e} = 1 $ & $\forall u\in \vertices$\\
    \cns\label{ilp:slmstart} & $x_e=x_d$ & for all sibling edges $e,d$\\
    \cns & $y_v + \idx{u}(1-x_{uv}) \geq y_u$ &$\forall uv \in \adjacencyedges\cup \extremityedges$\\
         & $\idx{u}(1-x_{uv})\geq y_u$& $\forall uv \in \selfedges$\\
    \cns\label{ilp:slmend} & $\idx{v}z_v \leq y_v$ & $\forall v\in \vertices\cup\pseudocaps $\\
\end{tabular}
\end{constraints}

\end{table}


\begin{table}

\begin{constraints}
\caption{Reporting for regular vertices.\comment{This is basically exactly the same as in ding, just swap out $r^{ab}$ for $t$ and $\parityvar$ for $r$ and root reports on vertices instead of edges.}} \label{tab:regv}
\begin{tabular}{lcl}
\cns\label{ilp:regvstart} & $\parityvar^\edt_v \leq 1 - x_{uv}$ & $\forall uv \in \selfedges^\A$\\
     & $\parityvar^\edt_v \geq  x_{uv}$ & $\forall uv \in \selfedges^\B$\\
\cns & $\parityvar_v \leq \parityvar_u +  (1-x_{uv})$& $\forall uv \in \extremityedges$\\
& $\parityvar_u \leq \parityvar_v + r_{uv}^{a\trans b} + (1-x_{uv})$& $\forall uv\in E_{adj}^\A,u,v\notin \pseudocaps$\\
& $\parityvar_u \leq \parityvar_v + (1-x_{uv})$& $\forall uv\in E_{adj}^\B,u,v\notin \pseudocaps$\\

\cns & $r_{v}^c \leq z_v$&$\forall v \in \vertices^\A$\\
\cns\label{ilp:regvend} & $r_{u}^{a\trans b} \leq x_{uv}$&$\forall uv\in\selfedges^\A$\\
\end{tabular}
\end{constraints}

\end{table}

\begin{table}

\begin{constraints}
\caption{Reporting for \gls{pcaps}.} \label{tab:pcaps}
\begin{tabular}{lcl}
\cns\label{ilp:pcstart} & $\parityvar_v = 0$ & $\forall v \in \pseudocaps^\A$\\
 & $\parityvar_v = 1$ & $\forall v \in \pseudocaps^\B$\\
\cns & $\parityvar_u \leq \parityvar_v + r_{v}^{A\trans B} + r_{v}^{A\trans b} + (1-x_{uv})$& $\forall uv\in \ades, v \in \pseudocaps^\A$\\
    & $\parityvar_u \leq \parityvar_v + r_{u}^{B\trans a} + (1-x_{uv})$& $\forall uv\in \ades, u\in \pseudocaps^\B$\\
    \cns & $r_{v}^{A\trans B} \leq z_v$&$\forall v \in \pseudocaps^\A$\\
    \cns & $1 -y_v \leq r^{A\trans b}_{v} + r^{A\cis a}_v$ & $v \in \pseudocaps^\A$\\
        & $1 -y_v \leq r^{B\trans a}_{v} + r^{B\cis b}_v$ & $v \in \pseudocaps^\B$\\
    \cns & $y_{v_i} \leq i(1-r_{v}^{R})$&$v\in \pseudocaps^\A,R\in  \{A\trans b, A\cis a\}$\\
     & $y_{v_i} \leq i(1-r_{v}^{R})$&$v\in \pseudocaps^\B,R\in  \{B\trans a, B\cis b\}$\\
    \cns\label{ilp:pcend} & $r_{v}^{A\trans B} \leq b_u$&$\forall uv\in \adjacencyedges, v \in \pseudocaps^A$ \\
    & $r_{v}^{A\trans b} \leq b_u$&$\forall uv\in \adjacencyedges, v \in \pseudocaps^A$ \\
    & $r_{v}^{B\trans a} \leq 1-b_u$&$\forall uv\in \adjacencyedges, v \in \pseudocaps^\B$\\
\end{tabular}
\end{constraints}

\end{table}

\begin{table} \caption{Reporting circular singletons.\comment{(This is what we developed last week.)}}\label{tab:csreport}
\begin{tabular}{lcl}
     \cns\label{ilp:csstart} & $d_u+d_v + x_{uv}\leq 2$ & $\forall uv \in \adjacencyedges\cup \selfedges$  \\
     & $d_u + d_v - x_{uv} \geq 0$ & $\forall uv \in \adjacencyedges\cup \selfedges$  \\
    \cns & $w_u = w_v$ & $\forall uv \in \selfedges$\\
    \cns\label{ilp:csend} &$K (1- x_{uv} + r^{s}_{u} + r^{s}_{v}) + w_v \geq w_u + d_v - d_u $ & $\forall uv\in \adjacencyedges$\\
     %& $K (1- x_{uv} + r^{s}_{v} + r^{s}_{u}) + w_u \geq w_v + d_u - d_v $ & $\forall uv\in \adjacencyedges$\\
\end{tabular}
\end{table}

\todoLeo{Fix inconsistencies with comparison in $\parityvar$ and $z$.}



\section{Evaluation}
\section{Discussion}



\subsubsection{Acknowledgements} Please place your acknowledgments at
the end of the paper, preceded by an unnumbered run-in heading (i.e.
3rd-level heading).

%
% ---- Bibliography ----
%
% BibTeX users should specify bibliography style 'splncs04'.
% References will then be sorted and formatted in the correct style.
%
\bibliographystyle{splncs04}
\bibliography{refs}


\end{document}
